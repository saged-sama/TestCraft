\documentclass[11pt]{article}
\usepackage{graphicx} % Required for inserting images
\usepackage{url}
\usepackage{hyperref} % Required for hyperlinks
\usepackage{xcolor} % Required for colorings

\hypersetup{
    colorlinks=true,
    linkcolor=black,
    urlcolor=green
}

% All new Custom Colors
\definecolor{javascriptColor}{RGB}{240, 210, 62}
\definecolor{htmlColor}{RGB}{255, 117, 25}
\definecolor{cssColor}{RGB}{0, 106, 235}
\definecolor{nodeColor}{RGB}{95, 166, 2}
\definecolor{reactColor}{RGB}{24, 199, 242}
\definecolor{mysqlColor}{RGB}{1, 105, 130}
\definecolor{sqliteColor}{RGB}{8, 82, 77}
\definecolor{vscodeColor}{RGB}{45, 163, 247}

% All New Custom Commands
\newcommand{\projectTitle}{\textbf{TestCraft}} % Find a nice Project Title Please
\newcommand{\documentationURL}{https://github.com/saged-sama/TestCraft/}
\newcommand{\JavaScript}{\href{https://www.javascript.com/}{\textbf{\colorbox{javascriptColor}{\textcolor{black}{JavaScript}}}}}
\newcommand{\HTML}{\href{https://html.com/}{\textbf{\textcolor{htmlColor}{HTML}}}}
\newcommand{\CSS}{\href{https://www.w3schools.com/css/}{\textbf{\textcolor{cssColor}{CSS}}}}
\newcommand{\NodeJS}{\href{https://nodejs.org/en}{\textbf{\textcolor{black}{N}\textcolor{nodeColor}{o}\textcolor{black}{de}\textcolor{nodeColor}{JS}}}}
\newcommand{\ReactJS}{\href{https://react.dev/}{\textbf{\textcolor{reactColor}{ReactJS}}}}
\newcommand{\ExpressJS}{\href{https://expressjs.com/}{\textbf{\textcolor{black}{Express}\colorbox{javascriptColor}{\textcolor{black}{JS}}}}}
\newcommand{\MySQL}{\href{https://www.mysql.com/}{\textbf{\textcolor{mysqlColor}{MySQL}}}}
\newcommand{\SQLite}{\href{https://www.sqlite.org/index.html}{\textbf{\textcolor{sqliteColor}{SQLite}}}}

\begin{document}
\begin{titlepage}
    \begin{center}
        \includegraphics[scale=0.10]{du.png}\par
        \begin{Huge}
            \textsc{University of Dhaka}\par
        \end{Huge}
        \begin{Large}
            Department of Computer Science and Engineering\par
            \vspace{1cm}
            CSE-3112 : Software Engineering Lab \\[12pt]
            Project Title: \projectTitle \\[12pt]
            Software Requirement Analysis Document
        \end{Large}
    \end{center}
    \vfill
    \begin{Large}
        \textbf{Developers:\\[12pt]}
        Name: Md Emon Khan \\[8pt]
        Roll: 30 \\[12pt]
        Name: Mahmudul Hasan \\[8pt]
        Roll: 60 \\[12pt]
    \end{Large}
\end{titlepage}
\newpage

% Do not Touch!! Table of Contents Go Here...
\tableofcontents
\newpage

\section{Introduction}
\projectTitle \space is a comprehensive online test designing and test-taking platform,
offering a solution for tutoring individuals and institutions alike. With
this web application, users can effortlessly create mock exams, practice tests,
or online assessments designed for their specific audience.

The web application offers some advanced features, including automated
assessment capabilities. This ensures a seamless and efficient evaluation process, saving time for both test creators
and participants. Additionally, \projectTitle \space provides automated notification
services, keeping users informed about important timelines, results, and
any pertinent updates from their associated institutions. This feature enhances
communication and ensures that all stakeholders stay well-informed
throughout the testing process.

    \subsection{Purpose of the System}
    The primary objective of \projectTitle \space is to deliver a user-friendly test creation
    tool tailored for individuals or institutions involved in tutoring. This
    empowers tutoring professionals to unwind and focus on personal aspects
    while efficiently managing their tutorship responsibilities.

    An equally crucial goal of our project is to afford students an easily affordable
    practice tool, enhancing their abilities. Students, hopefully, will be able to reflect on past assessments, get insights into their strengths and
    weaknesses for further development. Additionally, \projectTitle \space provides tutors
    with enhanced vision into the progress of each individual student they
    tutor.

    In essence, \projectTitle \space serves as a dynamic medium for interaction between
    educators and students, ensuring an optimized educational experience
    that yields the best possible outcomes for both.

    \subsection{Scope of the System}

    \subsection{Objective and Success Criteria of the System}
    The primary objective of the \projectTitle \space system is to provide a dynamic and interactive
    platform for educators and students which will enrich the educational experience by improving learning outcomes.
    
    Success criteria for the system include:
    \begin{itemize}
        \item Efficient test creation and administration
        \item Effortless assessment and feedback mechanisms
        \item Enhanced communication between tutors and students
        \item Improved student engagement and performance
    \end{itemize}

    \subsection{Definitions and Acronyms and Abbreviations}
    \begin{itemize}
        \item RBAC: Role Based Access Control
        \item MCQs: Multiple Choice Question
        \item RAD: Required Analysis Document
    \end{itemize}

    \subsection{References}

    \subsection{Overview}


\section{Overall Description}

    \subsection{Product Perspective}

    \subsection{Product Function}

    \subsection{User Profiles}

    \subsection{Constraint}

    \subsection{Assumption and Dependecies}



\section{Proposed System}

    \subsection{Overview}

    \subsection{Functional Requirement}

        \subsubsection{}

    \subsection{Non Functional Requirements}
        
        \subsubsection{Usability}
        \subsubsection{Reliability}
        \subsubsection{Performance}
        \subsubsection{Supportability}
        \subsubsection{Implementation}
        \subsubsection{Scalability}
        \subsubsection{Security}
        \subsubsection{Maintainability}
        \subsubsection{Testability}
    
    \subsection{System Models}

        \subsubsection{Scenarios}
        \subsubsection{Use Cases}
        \subsubsection{Use Case Model}
        \subsubsection{Dynamic Model}
            % \subsubsubsection{Sequence Diagram}
            % \subsubsubsection{Activity Diagram}
            % \subsubsubsection{State Diagram}
        \subsubsection{User Interface}
            % \subsubsubsection{User Interface}
            % \subsubsubsection{Software Interface}
            % \subsubsubsection{Hardware Interface}


\section{Supporting Information}

\end{document}
